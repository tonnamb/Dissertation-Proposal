\section{Introduction}

\subsection{Motivations}

The unique properties of metals nanocrystals are enabling state-of-art technologies.
Metal nanocrystals are clusters of metal atoms in a crystalline lattice with the size of 1 to 100 nanometers.
They are prevalent in applications ranging from catalysis \cite{astruc2008nanoparticles,Astruc_2006} to photovoltaics \cite{Atwater_2010}, DNA sequencing \cite{McNally_2010}, batteries \cite{Panniello_2014}, hydrogen storage \cite{Jena_2011,Ramos_Castillo_2015}, and cancer therapeutics \cite{Jain_2010,Kim_2010}.

Numerous properties of metal nanocrystals are found to depend on their size \cite{Roduner_2006} and shape \cite{Xia_2008}.
We are inspired to understand the governing phenomena that controls the size and shape of metal nanocrystals.
Through size- and shape-control, properties of metal nanocrystals can be tailored to enhance specific applications.
For instance, tetrahedral Pt nanocrystals are more active than spherical and cubic Pt nanocrystals as the catalyst for electron-transfer reactions \cite{Narayanan_2005}.
Higher proportions of surface atoms of a tetrahedron are on edges and corners, which is the reason for the higher catalytic activity.

\subsection{Scope}

We limit the scope of our investigation to shape-controlled synthesis of colloidal metal nanocrystals.
Metal nanocrystals can be produced through vapor-phase \cite{Swihart_2003} or colloidal methods \cite{Tao_2008}.
Solution-phase syntheses are more powerful and versatile for generating metal nanocrystals of different shapes.
In addition, colloidal methods can synthesize large quantities of nanocrystals with matured solution-phase processing technology, without the specialized equipments needed for vapor-phase methods.

Well-defined shapes of metal nanocrystals such as cube \cite{Im_2005}, octahedron \cite{Xia_2012}, icosahedron \cite{Xiong_2007}, triangular plate \cite{Lofton_2005} and five-fold twinned pentagonal wire \cite{Tsuji_2008} can be synthesized in the colloidal system.
A comprehensive review of shape-controlled solution-phase synthesis of metal nanocrystals is given by Xia et al. \cite{Xia_2008,Xia_2015}.
Although there numerous efforts to solidify the nucleation and growth mechanisms of these syntheses \cite{Lofton_2005,Mariscal_2012,Park_2013,Viswanath_2009,Liao_2014,Chang_2011,Murph_2015}, these mechanisms still remain elusive due to the lack of atomic resolution and quantitative data. This proposal describes how atomistic-scale simulation can be used to probe the nucleation and growth mechanisms.

\subsection{Research objective}

\subsection{Significance}
    
    
    