% Outline given by http://theprofessorisin.com/2011/07/05/dr-karens-foolproof-grant-template/

\section{Introduction}

% Large general topic of wide interest

Nanoscale materials have the potential to solve many of today's biggest problems such as peak oil, the global water crisis, and the burden of cancer.
Faraday's discovery of colloidal ruby gold producing different colored solutions in 1857 \cite{Faraday_1857,Thompson_2007} has inspired generations of nanoscale science.
Controlled synthesis of nanocrystals, namely quantum dots, was invented by Bawendi \textit{et al.} in 1993 \cite{hakimi1993quantum,Murray_2000}.
They paved the way for the utilization of well-defined nanocrystals in various fields ranging from heterogeneous catalysis \cite{astruc2008nanoparticles,Astruc_2006} to photovoltaics \cite{Atwater_2010}, DNA sequencing \cite{McNally_2010}, batteries \cite{Panniello_2014}, hydrogen storage \cite{Jena_2011,Ramos_Castillo_2015}, and cancer therapeutics \cite{Jain_2010,Kim_2010}.

% Two bodies of literature pointing to a debate
% Experimental
% Computational
Nanocrystals can be grown to specific sizes and shapes, but the question of what is the growth controlling mechanism remains elusive.
This is important because numerous properties of metal nanocrystals are found to depend on their size \cite{Roduner_2006} and shape \cite{Xia_2008}.
In catalysis for instance, tetrahedral Pt nanocrystals are more active than spherical and cubic Pt nanocrystals as the catalyst for electron-transfer reactions \cite{Narayanan_2005}.
The controlling mechanism for various systems have been studied using both experimental and computational techniques.
Experimentalists have employed techniques such as \textit{in situ} transmission electron microscopy \cite{Liao_2014,Woehl_2014}, X-ray photoelectron spectroscopy \cite{Gao_2004,Park_2014,Huang_1996,Kedia_2012,Bonet_2000}, and variation of reaction parameters \cite{Personick_2013,Xia_2012,Zeng_2010,Zhang_1996,Chang_2011,Zhu_2011} to elucidate the controlling mechanism.
Theorists address this important question using techniques such as density functional theory \cite{Kilin_2008,Al_Saidi_2012,Saidi_2013,Zhang_2008} and molecular dynamics \cite{Zhou_2014}.

% However
However, previous work in the literature have not yet adequately addressed the atomic scale mechanism of shape control in the solution phase.
% Gap in knowledge
Despite much excellent experimental and theoretical work, theorists examining the mechanism have not yet explored the influence of the solution phase in growing nanocrystals with well-defined shapes.
% Urgency
Yet, without such an understanding, we are left with an incomplete description of the mechanism of shape control that creates the condition for ill-informed reaction engineering for scale-up.
To date, only a few syntheses have been scaled up to the gram-scale, and yet they still have poor quality control \cite{Jana_2005,Lohse_2013}.
% Hero narrative
This study will remedy this gap in the literature by examining how shape control is achieved in nanocrystals synthesis using molecular dynamics simulation, in which explicit solvent is computationally feasible and observations in the atomic resolution can be made.


  
  
  