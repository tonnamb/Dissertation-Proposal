% Outline given by http://theprofessorisin.com/2011/07/05/dr-karens-foolproof-grant-template/

\section{Introduction}

% Large general topic of wide interest

Nanoscale materials have the potential to solve many of today's biggest problems such as peak oil, the global water crisis, and the burden of cancer.
Faraday's discovery of colloidal ruby gold producing different colored solutions in 1857 \cite{Faraday_1857,Thompson_2007} has inspired generations of nanoscale science.
Controlled synthesis of nanocrystals, namely quantum dots, was invented by Bawendi \textit{et al.} in 1993 \cite{hakimi1993quantum,Murray_2000}.
They paved the way for the utilization of well-defined nanocrystals in various fields ranging from heterogeneous catalysis \cite{astruc2008nanoparticles,Astruc_2006} to photovoltaics \cite{Atwater_2010}, DNA sequencing \cite{McNally_2010}, batteries \cite{Panniello_2014}, hydrogen storage \cite{Jena_2011,Ramos_Castillo_2015}, and cancer therapeutics \cite{Jain_2010,Kim_2010}.

% Two bodies of literature pointing to a debate
% TEM studies
% Mechanistic studies

Nanocrystals can be grown to specific sizes and shapes, but the question of what is the controlling mechanism remains elusive.
This is important because numerous properties of metal nanocrystals are found to depend on their size \cite{Roduner_2006} and shape \cite{Xia_2008}.
In catalysis for instance, tetrahedral Pt nanocrystals are more active than spherical and cubic Pt nanocrystals as the catalyst for electron-transfer reactions \cite{Narayanan_2005}.
Scholars have studied the controlling mechanism for various systems using experimental and computational techniques.
