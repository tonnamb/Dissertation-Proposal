\section{Introduction}

\subsection{Significance}

The unique properties of metals nanocrystals are enabling state-of-art technologies.
Metal nanocrystals are clusters of metal atoms in a crystalline lattice with the size of 1 to 100 nanometers.
They are prevalent in applications ranging from heterogeneous catalysis \cite{astruc2008nanoparticles,Astruc_2006} to photovoltaics \cite{Atwater_2010}, DNA sequencing \cite{McNally_2010}, batteries \cite{Panniello_2014}, hydrogen storage \cite{Jena_2011,Ramos_Castillo_2015}, and cancer therapeutics \cite{Jain_2010,Kim_2010}.

Numerous properties of metal nanocrystals are found to depend on their size \cite{Roduner_2006} and shape \cite{Xia_2008}.
We are inspired to understand the governing phenomena that controls the shape of metal nanocrystals.
Through shape-control, properties of metal nanocrystals can be tailored to enhance specific applications.
For instance, tetrahedral Pt nanocrystals are more active than spherical and cubic Pt nanocrystals as the catalyst for electron-transfer reactions \cite{Narayanan_2005}.
Higher proportions of surface atoms of a tetrahedron are on edges and corners, which is the reason for the higher catalytic activity.

\subsection{Scope}

We limit the scope of our investigation to synthesis of colloidal metal nanocrystals.
Metal nanocrystals can be produced through vapor-phase \cite{Swihart_2003} and colloidal methods \cite{Tao_2008}. Examples of vapor-phase methods are inert gas condensation \cite{Wegner_2002,Simchi_2007}, chemical vapor synthesis \cite{Lee_2012,Ostraat_2001}, and flame spray pyrolysis \cite{Teoh_2010}. 
Vapor-phase methods require a high temperature (over 1000 $^{\circ}$C \cite{Smetana_2005}), vacuum and expensive equipments.
The colloidal method typically involves a metal salt precursor reduced in solution in the presence of a dispersant, which prevents the aggregation of nanoparticles and improve their stability.
Advantages of the colloidal method are: 
1) No specialized equipment required; 
2) Solution-based processing is matured and readily available
3) Parameters such as solvent, temperature, precursor concentration, surface capping and foreign species can be tweaked to make versatile nanocrystals.
4) Continuous process allow large yields of nanocrystals to be synthesized.

Shape-control in noble metal systems, Ag in particular, is our interest.
This is because the structure-property relationship of Ag nanocrystals is evident, particularly in their capabilities in localized surface plasmon resonance.
That is when light incident upon the surface, collective oscillation of free electrons (also known as plasmons) arises as light wave are trapped within nanocrystals smaller than the wavelength of light \cite{Petryayeva_2011}.
The geometric shape of the nanocrystal dictates the number of resonance peaks, their wavelengths, and the partitioning between scattering and absorption cross sections.
This is because polarization of free electrons and charges distribution occurs over the nanocrystal surface.
Among all metals, Ag exhibits the strongest plasmonic interaction with light \cite{Lu_2009}.
Achieving tight shape-control can complement applications such as bioassays based on surface-enhanced Raman spectroscopy (SERS).
For example, a localized surface plasmon resonance bio-chip can be used for real-time detection of insulin \cite{Hiep_2008}.

\subsection{Need}

Well-defined shapes of Ag nanocrystals such as cube \cite{Im_2005}, octahedron \cite{Xia_2012}, icosahedron \cite{Xiong_2007}, triangular plate \cite{Xiong_2007} and five-fold twinned pentagonal wire \cite{Tsuji_2008} can be synthesized with colloidal methods.
Although a variety of Ag nanocrystal shapes can be synthesized, the shape-control mechanism still remains elusive.
Previous efforts to solidify the mechanism of shape-control will be described in the literature review section.
This knowledge is critical in designing robust methods for large-scale production of nanocrystals with well-defined shape.
Consequently, large-scale production enables the commercialization of nanocrystals products with outstanding properties that exceeds what is available in the current market.
  