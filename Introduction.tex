\section{Introduction}

\subsection{Motivations}

The unique properties of metals are enabling numerous state-of-art technologies.
They are essential in applications ranging from catalysis \cite{astruc2008nanoparticles,Astruc_2006} to photovoltaics \cite{Atwater_2010}, photonics, information storage, electronics, energy storage, environmental protection, and cancer therapeutics \cite{Jain_2010,Kim_2010}.


Nanoparticles are clusters of atoms or molecules with the size of 1 to 100 nanometers. 
Nanocrystals are nanoparticles with a single crystalline lattice. 
Numerous properties of nanocrystals are found to depend on their size \cite{Roduner_2006} and shape \cite{Xia_2008}. 
The ability to control the size and shape of nanocrystals allows their properties to be tailored towards applications such as catalysis \cite{astruc2008nanoparticles,Astruc_2006}, solar cell \cite{Atwater_2010}, and cancer therapeutics \cite{Jain_2010,Kim_2010}. 

For instance, tetrahedral Pt nanocrystals are more active than spherical and cubic Pt nanoparticles as the catalyst for electron-transfer reactions \cite{Narayanan_2005}. Higher proportion of surface atoms of tetrahedrals nanoparticles are on edges and corners, which accounts for the higher catalytic activity. 

Drug delivery shape.

\subsection{Scope}

Well-defined shapes of metal nanoparticles such as cube \cite{Im_2005}, octahedron \cite{Xia_2012}, icosahedron \cite{Xiong_2007}, triangular plate \cite{Lofton_2005} and five-fold twinned pentagonal wire \cite{Tsuji_2008} can be synthesized in the colloidal system. A comprehensive review of shape-controlled solution-phase synthesis of metal nanoparticles is given in the reference \cite{Xia_2008}. Although there numerous efforts to solidify the nucleation and growth mechanisms of these syntheses \cite{Lofton_2005,Mariscal_2012,Park_2013,Viswanath_2009,Liao_2014,Chang_2011,Murph_2015}, these mechanisms still remain elusive due to the lack of atomic resolution and quantitative data. This proposal describes how atomistic-scale simulation can be used to probe the nucleation and growth mechanisms.

\subsection{Research objective}

\subsection{Significance}
    