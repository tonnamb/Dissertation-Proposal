\section{Literature review}

Shape-control in noble metal systems, Ag in particular, is our interest.
This is because the structure-property relationship of Ag nanocrystals is evident, particularly in their capabilities in localized surface plasmon resonance.
That is when light incident upon the surface, collective oscillation of free electrons (also known as plasmons) arises as light wave are trapped within nanocrystals smaller than the wavelength of light \cite{Petryayeva_2011}.
The geometric shape of the nanocrystal dictates the number of resonance peaks, their wavelengths, and the partitioning between scattering and absorption cross sections.
This is because polarization of free electrons and charges distribution occurs over the nanocrystal surface.
Among all metals, Ag exhibits the strongest plasmonic interaction with light \cite{Lu_2009}.
Achieving tight shape-control can complement applications such as bioassays based on surface-enhanced Raman spectroscopy (SERS).
For example, a localized surface plasmon resonance bio-chip can be used for real-time detection of insulin \cite{Hiep_2008}.

Metal nanocrystals can be produced through vapor-phase \cite{Swihart_2003} and colloidal methods \cite{Tao_2008}. Examples of vapor-phase methods are inert gas condensation \cite{Wegner_2002,Simchi_2007}, chemical vapor synthesis \cite{Lee_2012,Ostraat_2001}, and flame spray pyrolysis \cite{Teoh_2010}. 
Vapor-phase methods require a high temperature (over 1000 $^{\circ}$C \cite{Smetana_2005}), vacuum and expensive equipments.
The scope of my investigation is limited to the synthesis of colloidal metal nanocrystals.
The colloidal method typically involves a metal salt precursor reduced in solution in the presence of a dispersant, which prevents the aggregation of nanoparticles and improve their stability.
Advantages of the colloidal method are: 
1) No specialized equipment required; 
2) Solution-based processing is matured and readily available;
3) Parameters such as solvent, temperature, precursor concentration, surface capping and foreign species can be tweaked to make versatile nanocrystals;
4) Continuous process allow large yields of nanocrystals to be synthesized.