\section{Literature review}

The unique properties of metals nanocrystals are enabling state-of-art technologies.
Metal nanocrystals are clusters of metal atoms in a crystalline lattice with the size of 1 to 100 nanometers.
Shape-control in noble metal systems, Ag in particular, is our interest.
This is because the structure-property relationship of Ag nanocrystals is evident, particularly in their capabilities in localized surface plasmon resonance.
That is when light incident upon the surface, collective oscillation of free electrons (also known as plasmons) arises as light wave are trapped within nanocrystals smaller than the wavelength of light \cite{Petryayeva_2011}.
The geometric shape of the nanocrystal dictates the number of resonance peaks, their wavelengths, and the partitioning between scattering and absorption cross sections.
This is because polarization of free electrons and charges distribution occurs over the nanocrystal surface.
Among all metals, Ag exhibits the strongest plasmonic interaction with light \cite{Lu_2009}.
Achieving tight shape-control can complement applications such as bioassays based on surface-enhanced Raman spectroscopy (SERS).
For example, a localized surface plasmon resonance bio-chip can be used for real-time detection of insulin \cite{Hiep_2008}.

Metal nanocrystals can be produced through vapor-phase \cite{Swihart_2003} and colloidal methods \cite{Tao_2008}. Examples of vapor-phase methods are inert gas condensation \cite{Wegner_2002,Simchi_2007}, chemical vapor synthesis \cite{Lee_2012,Ostraat_2001}, and flame spray pyrolysis \cite{Teoh_2010}. 
Vapor-phase methods require a high temperature (over 1000 $^{\circ}$C \cite{Smetana_2005}), vacuum and expensive equipments.
The scope of my investigation is limited to the chemical synthesis of colloidal metal nanocrystals.
The colloidal method typically involves a metal salt precursor reduced in solution in the presence of a dispersant, which prevents the aggregation of nanoparticles and improve their stability.
Advantages of the colloidal method are: 
1) No specialized equipment required; 
2) Solution-based processing is matured and readily available;
3) Parameters such as solvent, temperature, precursor concentration, surface capping and foreign species can be tweaked to make versatile nanocrystals;
4) Continuous process allow large yields of nanocrystals to be synthesized.

The growth of colloidal nanocrystals can be roughly divided into three stages \cite{Xia_2008}:
1) nucleation, where individual metal atoms cluster together to form nuclei;
2) evolution of nuclei into seeds;
3) growth of seeds into nanocrystals.
The difference between nuclei and seeds is the structure of nuclei can fluctuate but the seeds cannot.
In a thermodynamic-control growth, the seed shape is known to define the nanocrystal shape.
Although the process can be roughly defined, our current understanding of the evolution pathway is far from being able to visualize the atomistic details of how seeds nucleate and evolve into nanocrystals of specific shape.

Colloidal methods to synthesize Ag nanocrystals include citrate reduction \cite{Wu_2008,Lee_1982}, silver mirror reaction \cite{Yin_2002}, polyol synthesis \cite{Wiley_2008,Sun_2002}, seed-mediated growth \cite{Pietrobon_2009,Sun_2002,Zhang_2010}, and light-mediated synthesis \cite{Pietrobon_2008,Jin_2003,Zhou_2008}.
The polyol synthesis is a common and successful synthesis route, in which nanocrystals with well-defined shapes and sizes can be obtained.
% Talk about different polyol synthesis

Slight modifications of the polyol synthesis protocol, such as introducing foreign species, enable different nanocrystal shapes to be produced.
In the synthesis of Ag nanocubes, trace amount of Na_2S can be added to increase the selectivity towards nanocubes \cite{Skrabalak_2007}.
The addition of Na_2S allows the formation of Ag_2S nanocrystals, which catalyzes the reduction of AgNO_3.
It is proposed that the faster reduction process can limit formation of twinned Ag seeds \cite{Wiley_2006}, thus favoring nanocube formations.
Comparatively, Ag nanowires can be synthesized by adding NaCl or KCl as a source of Cl$^-$ ions \cite{Tsuji_2008}.
Their experiments have shown that Cl$^-$ ions can accelerate dissolution of spherical nanoparticles, which promotes the growth of one-dimensional nanocrystals.
As a final example, Ag nanoplates can be formed by substituting PVP with polyacrylamide into the polyol synthesis \cite{Xiong_2007}.
It has been reported that the amino groups of polyacrylamide can form complexes with metal cations \cite{Sari_2006}, greatly reducing the potential of the Ag/Ag$^+$ pair, thus reduction rate of AgNO_3 is significantly reduced.
The reduced reduction rate favors the formation of nanoplates through kinetic control.

Seed-mediated growth is essentially the polyol synthesis with nanoparticle seeds added to disentangle growth from nucleation.
This allows enhanced control of size and shape.
% Talk about different seeded growth
  