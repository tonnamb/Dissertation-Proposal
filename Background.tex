\section{Background}

The unique properties of metals nanocrystals are enabling state-of-art technologies.
Metal nanocrystals are clusters of metal atoms in a crystalline lattice with the size of 1 to 100 nanometers.
Shape-control in noble metal systems, Ag in particular, is our interest.
This is because the structure-property relationship of Ag nanocrystals is evident, particularly in their capabilities in localized surface plasmon resonance.
That is when light incident upon the surface, collective oscillation of free electrons (also known as plasmons) arises as light wave are trapped within nanocrystals smaller than the wavelength of light \cite{Petryayeva_2011}.
The geometric shape of the nanocrystal dictates the number of resonance peaks, their wavelengths, and the partitioning between scattering and absorption cross sections.
This is because polarization of free electrons and charges distribution occurs over the nanocrystal surface.
Among all metals, Ag exhibits the strongest plasmonic interaction with light \cite{Lu_2009}.
Achieving tight shape-control can complement applications such as bioassays based on surface-enhanced Raman spectroscopy (SERS).
For example, a localized surface plasmon resonance bio-chip can be used for real-time detection of insulin \cite{Hiep_2008}.

\subsection{Synthesis of Metal Nanocrystals}

Metal nanocrystals can be produced through vapor-phase \cite{Swihart_2003} and colloidal methods \cite{Tao_2008}. Examples of vapor-phase methods are inert gas condensation \cite{Wegner_2002,Simchi_2007}, chemical vapor synthesis \cite{Lee_2012,Ostraat_2001}, and flame spray pyrolysis \cite{Teoh_2010}. 
Vapor-phase methods require a high temperature (over 1000 $^{\circ}$C \cite{Smetana_2005}), vacuum and expensive equipments.
The scope of my investigation is limited to the chemical synthesis of colloidal metal nanocrystals.
The colloidal method typically involves a metal salt precursor reduced in solution in the presence of a dispersant, which prevents the aggregation of nanoparticles and improve their stability.
Advantages of the colloidal method are: 
1) No specialized equipment required; 
2) Solution-based processing is matured and readily available;
3) Parameters such as solvent, temperature, precursor concentration, surface capping and foreign species can be tweaked to make versatile nanocrystals;
4) Continuous process allow large yields of nanocrystals to be synthesized.

The growth of colloidal nanocrystals can be roughly divided into three stages \cite{Xia_2008}:
1) nucleation, where individual metal atoms cluster together to form nuclei;
2) evolution of nuclei into seeds;
3) growth of seeds into nanocrystals.
The difference between nuclei and seeds is the structure of nuclei can fluctuate but the seeds cannot.
In a thermodynamic-control growth, the seed shape is known to define the nanocrystal shape.
Although the process can be roughly defined, our current understanding of the evolution pathway is far from being able to visualize the atomistic details of how seeds nucleate and evolve into nanocrystals of specific shape.

Colloidal methods to synthesize Ag nanocrystals include citrate reduction \cite{Wu_2008,Lee_1982}, silver mirror reaction \cite{Yin_2002}, polyol synthesis \cite{Wiley_2008,Sun_2002}, seed-mediated growth \cite{Pietrobon_2009,Sun_2002,Zhang_2010}, and light-mediated synthesis \cite{Pietrobon_2008,Jin_2003,Zhou_2008}.
The polyol synthesis is a common and successful synthesis route, in which nanocrystals with well-defined shapes and sizes can be obtained.
% Talk about self-seeding polyol synthesis
In a self-seeding polyol synthesis, $AgNO_3$ and PVP are dissolved in ethylene glycol.
The mixture is typically refluxed at 150$^{\circ}$C.
The morphology and dimensions of the product were shown to depend on different aspects of the reaction conditions \cite{Sun_2002}.
Irregular nanoparticles were obtained when the temperature is below 120$^{\circ}$C or above 190$^{\circ}$C.
At the initial concentration of $AgNO_3$ higher than 0.1 M nanocubes were formed, otherwise nanowires were the major product.
Multiply twinned particles were obtained instead of single crystal cubes when the molar ratio of PVP to $AgNO_3$ was increased from 1.5 to 3.
It was shown that nanowires with higher aspect ratios can be obtained by reducing the molar ratio of PVP to $AgNO_3$ \cite{Sun_2002}.
The optimal molar ratio was found to be 1.5:1, noting that the concentration of $AgNO_3$ is below 0.1 M.
Although at a molar ratio of less than 1:1, the yield and monodispersity of nanowires were too unacceptably low.
Higher degree of polymerization of PVP can also promote the formation of nanowires with high aspect ratios \cite{Sun_2002}.
Irregular nanoparticles are formed when 1-ethyl-2-pyrrolidone, the monomer of PVP, is used in place of PVP.
This observation demonstrates the importance of the steric effect of PVP that prevents aggregation between nanoparticles.

% Introducing foreign species
Modifications of the polyol synthesis protocol by introducing foreign species enables a highly selective production of different nanocrystal shapes.
In the synthesis of Ag nanocubes, trace amount of $Na_2S$ can be added to increase the selectivity towards nanocubes \cite{Skrabalak_2007,Siekkinen_2006}.
The addition of $Na_2S$ allows the formation of $Ag_2S$ nanocrystals, which catalyzes the reduction of $AgNO_3$.
It is proposed that the faster reduction process can limit formation of twinned Ag seeds \cite{Wiley_2006}, thus promoting the formation of nanocubes.
% Comparatively, monodisperse Ag nanowires can be synthesized by adding NaCl or KCl as a source of Cl$^-$ ions \cite{Tsuji_2008}.
% Their experiments have shown that Cl$^-$ ions can accelerate dissolution of spherical nanoparticles, which promotes the growth of one-dimensional nanocrystals.
Ag nanoplates can be obtained with yields as high as 90 \% by substituting PVP with polyacrylamide \cite{Xiong_2007}.
It has been reported that the amino groups of polyacrylamide can form complexes with metal cations \cite{Sari_2006}, greatly reducing the potential of the Ag/Ag$^+$ pair, thus reduction rate of $AgNO_3$ is significantly reduced.
The reduced reduction rate favors the formation of nanoplates through kinetic control.
Introducing Fe(II) ions into the system allows for the control over oxidative etching \cite{Wiley_2005}.
The Fe(II) ions can remove adsorbed oxygen atoms from the surface of Ag nanocrystals and prevent their dissolution by oxidative etching.
At a high concentration of Fe(II) ions, nanowires were obtained because dissolution of twinned seeds were inhibited.
On the other hand, single-crystal nanocubes formation was favored at low concentration of Fe(II) ions because adsorbed oxygen atoms were only partially removed, and thus twinned seeds were selectively etched.
It was proposed that replacing $AgNO_3$ with $CF_3COOAg$ as the new silver precursor can improve both the robustness and reproducibility of the polyol synthesis \cite{Zhang_2010}.
The argument was that the trifluoroacetate group is more stable than the nitrate group.
The nitrate group may decompose at an elevated temperature, making the synthesis more difficult to understand and control.
High-quality Ag nanocubes with controllable dimensions were produced with $CF_3COOAg$ in the presence of NaSH and HCl.

% Talk about different seeded growth
Seed-mediated growth is essentially the polyol synthesis with nanoparticle seeds added to disentangle growth from nucleation.
This method allows for a greater control over the final morphology.
The two main categories of seed-mediated growth are homoepitaxial and heteroepitaxial growth.
For a homoepitaxial growth, the seed crystal is composed of the same metal as the atoms being deposited.
With either spherical or cubic seeds, Ag nanocubes with edge lengths controllable in the range of 30 to 200 nm can be synthesized \cite{Zhang_2010}.
Tunable reaction parameters, additional to the self-seeding process, include molar ratio of Ag seeds to $AgNO_3$ and concentration of Ag seeds added.
Remarkable size control can also be achieved as demonstrated in the seed-mediated synthesis of Ag decahedrons \cite{Pietrobon_2008}.
Seeds of Ag decahedral particles can be grown in a controllable fashion by mixing in additional precursor solution and exposing the mixture to white light of a metal halide lamp for 20 h.
These regrowth steps can be repeated to produce Ag decahedrons ranging from 40 to 120 nm.
The decoupling of growth from nucleation allow the nanocrystal size to be controlled while fully preserving the monodispersity.
For a heteroepitaxial growth, the seeds and the deposited atoms are chemically different.
Difference in lattice constants between the two types of metals plays an important role in heteroepitaxial growth.
For example, Au and Ag have a lattice mismatch of only 0.25\% therefore Au seeds have been successfully used as template for Ag deposition.
Synthesized shapes with the Au-Ag core-shell structure include triangular bifrustums \cite{Yoo_2009} and nanorods \cite{Seo_2008,Tsuji_2006}.
When there is a large lattice mismatch, anisotropic growth are promoted because isotropic growth is inhibited by high strain energy.
For example, Pt and Ag have a lattice mismatch of 4.15\% thus Pt nanocrystals can be used as seeds for Ag nanowires growth \cite{Sun_2002,Sun_2002b,Tsuji_2008b}.
  
  
  