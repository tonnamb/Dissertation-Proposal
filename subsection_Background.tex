\subsection{Goal and Objectives}

We aim to develop a fundamental understanding of how Ag nanocrystals are grown to specific shapes, such as nanocubes \cite{Im_2005}, nanoplates \cite{Lofton_2005} and nanowires \cite{Tsuji_2008}.
Using molecular simulations, we hope to answer these intriguing questions:
How are {100} facets stabilized in the growth of nanocubes?
How are twinned nanocrystals formed and stabilized?
How does structure-directing agents, foreign species, and solvent molecules contribute to shape selectivity?
Our ultimate goal is to provide guidelines for engineers on designing syntheses that produces batches of mono-disperse metal nanocrystals.

With the goals described in mind, our research objectives:
\begin{enumerate}
\item Compute the solution-phase binding energies of PVP to Ag(100) and Ag(111) surfaces to elucidate the role of structure-directing agents in shape-controlled synthesis.
\item Compare rate of Ag atom addition to Ag surfaces that are adsorbed by PVP to verify that \{100\} faceted nanocrystals can be favored by kinetic control growth.
\item Measure the influence of 2-pyrrolidone on PVP layers adsorbed on Ag surfaces to explain truncation of nanocubes in the presence of 2-pyrrolidone.
\item Quantify the interaction between Ag nanoplates in the oriented attachment process to clarify the mechanism of the process.
\item Calculate energy profiles of shape-specific seed formation in different environments to understand how kinetic and thermodynamic influence seed formation.
\end{enumerate}

  
  
  
  
  
  
  
  
  
  