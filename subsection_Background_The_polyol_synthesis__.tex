\subsection{Background}

The polyol synthesis is the most popular solution-phase synthesis of Ag nanocrystals \cite{Skrabalak_2007}.
In the polyol synthesis, ethylene glycol acts as both the solvent and the reducing agent.
The source of Ag is from AgNO_3 that is dissolved in ethylene glycol.
Structure-directing agents, typically polyvinylpyrrolidone (PVP), are added to prevent aggregation of nanocrystals and direct their shape.
The ratio of PVP to AgNO_3 is critical to the formation of different nanocrystal shapes.
The typical reaction temperature is $150^{\circ}$ C, although it can vary for different nanocrystal shapes with different protocols.

Slight modifications of the polyol synthesis protocol, such as introducing foreign species, enable different nanocrystal shapes to be produced.
In the synthesis of Ag nanocubes, trace amount of Na_2S can be added to increase the selectivity towards nanocubes \cite{Skrabalak_2007}.
The addition of Na_2S allows the formation of Ag_2S nanocrystals, which catalyzes the reduction of AgNO_3.
It is proposed that the faster reduction process can limit formation of twinned Ag seeds \cite{Wiley_2006}, thus favoring nanocube formations.
Comparatively, Ag nanowires can be synthesized by adding NaCl or KCl as a source of Cl^- ions \cite{Tsuji_2008}.
Their experiments have shown that Cl^- ions can accelerate dissolution of spherical nanoparticles, which promotes the growth of one-dimensional nanocrystals.

\subsection{Goal and Objectives}

We aim to gain a fundamental understanding of how Ag nanocrystals are grown to specific shapes, such as nanocubes \cite{Im_2005}, nanoplates \cite{Lofton_2005} and nanowires \cite{Tsuji_2008}.
Using molecular simulations, we hope to answer these intriguing questions:
How are {100} facets stabilized in the growth of nanocubes?
How are twinned nanocrystals formed and stabilized?
How does structure-directing agents, foreign species, and solvent molecules contribute to shape selectivity?
Our ultimate goal is to provide guidelines for engineers on designing syntheses that produces batches of mono-disperse metal nanocrystals.

With the goals described in mind, our research objectives:
\begin{enumerate}
\item Compute the solution-phase binding energies of PVP to Ag(100) and Ag(111) surfaces to elucidate the role of structure-directing agents in shape-controlled synthesis.
\item Compare rate of Ag atom addition to Ag surfaces that are adsorbed by PVP to verify that \{100\} faceted nanocrystals can be favored by kinetic control growth.
\item Measure the influence of 2-pyrrolidone on PVP layers adsorbed on Ag surfaces to explain truncation of nanocubes in the presence of 2-pyrrolidone.
\item Quantify the interaction between Ag nanoplates in the oriented attachment process to clarify the mechanism of the process.
\item Calculate energy profiles of shape-specific seed formation in different environments to understand how kinetic and thermodynamic influence seed formation.
\end{enumerate}

  