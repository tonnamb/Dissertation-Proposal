\subsection{Mechanistic Studies of Shape Control}

Background:

Xia 2015 Kinetics Thermodynamics

Nanocrystals synthesis can be viewed as a set of sequential and parallel chemical reactions.
Each route of chemical reactions yields nanocrystals of different dimensions and morphology.
The end product of chemical reactions can be controlled by a set of thermodynamic and kinetic parameters.
For the synthesis of metal nanocrystals, thermodynamic parameters can include reduction potential and surface capping.
The kinetic parameters can include concentration, mass transport, temperature, and the involvement of foreign species.
These parameters dictates the shape of the free energy landscape and how well the system can minimize its free energy.
The free energy landscape can be depicted by wells of free energy minima separated by free energy barriers, as shown in Figure \ref{fig:energy-landscape}.
The thermodynamic product can be obtained if the system can overcome all free energy barriers and reach the global minimum.
This is ideally achievable at high temperature or with long relaxation time after each sequential reaction.
When the system is stuck in a local minimum, the kinetic product is obtained.
In this case, the system do not have enough time or energy to cross the free energy barriers for the system to be at the global minimum.
We can see that temperature plays an important role in determining whether the synthesis is controlled by thermodynamics or kinetics.
It is interesting to point out that the thermodynamic product only depends on the final state, while the kinetic product depends on the exact process.

The growth process of metal nanocrystals consists of atoms initially adding to a specific site on the nanocrystal surface (atom deposition) and migrating to the site lowest in surface free energy (surface diffusion).
It is energetically favorable for the atoms to deposit at the most active site of the nanocrystal surface with the highest surface free energy.
The nanocrystal shape will be determined by the relative magnitudes of the atom deposition rate (V_{deposition}) and the surface diffusion rate (V_{diffusion}).
The relative magnitudes can be concisely represented by the ratio between the rates for atom deposition and surface diffusion (V_{deposition}/V_{diffusion}).
Xia et al. have comprehensively reviewed the perspective between thermodynamic and kinetic products in the shape-controlled synthesis of colloidal metal nanocrystals \cite{Xia_2015}.

Xia 2013 Surface diffusion

Experiments:

Cubes

Wires

PVP facet selectivity

PVP interactions

Bridge Tang 2008 Strain-induced

Simulations:

Fichthorn DFT

Fichthorn Force field

Solvent effects in experiments

Fichthorn solvent
  
  
  
  