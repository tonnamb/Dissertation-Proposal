\subsection{Mechanistic Studies of Shape Control}

Background: Xia 2015 Kinetics Thermodynamics. Xia 2013 Surface diffusion.

Nanocrystals synthesis can be viewed as a set of sequential and parallel chemical reactions.
Each route of chemical reactions yields nanocrystals of different dimensions and morphology.
The end product of chemical reactions can be controlled by a set of thermodynamic and kinetic parameters.
For the synthesis of metal nanocrystals, thermodynamic parameters can include reduction potential and surface capping.
The kinetic parameters can include concentration, mass transport, temperature, and the involvement of foreign species.
These parameters dictates the shape of the free energy landscape and how well the system can minimize its free energy.
The free energy landscape can be depicted by wells of free energy minima separated by free energy barriers, as shown in Figure \ref{fig:energy-landscape}.
The thermodynamic product can be obtained if the system can overcome all free energy barriers and reach the global minimum.
This is ideally achievable at high temperature or with long relaxation time after each sequential reaction.
When the system is stuck in a local minimum, the kinetic product is obtained.
In this case, the system do not have enough time or energy to cross the free energy barriers for the system to be at the global minimum.
We can see that temperature plays an important role in determining whether the synthesis is controlled by thermodynamics or kinetics.
It is interesting to point out that the thermodynamic product only depends on the final state, while the kinetic product depends on the exact process.

Xia et al. have comprehensively reviewed the perspective between thermodynamic and kinetic products in the shape-controlled synthesis of colloidal metal nanocrystals \cite{Xia_2015}.
The growth process of metal nanocrystals consists of atoms initially adding to a specific site on the nanocrystal surface (atom deposition) and migrating to the site lowest in surface free energy (surface diffusion).
Atoms tend to deposit at the most active region with the highest surface free energy, such as corners, edges, and high-index facets.
This is because it is energetically favorable to stabilize the most active sites of the nanocrystal surface.
The shape of the nanocrystal will be determined by the relative magnitudes of the atom deposition rate (V_{deposition}) and the surface diffusion rate (V_{diffusion}).
The relative magnitudes can be concisely represented by the ratio between the rates for atom deposition and surface diffusion (V_{deposition}/V_{diffusion}) \cite{Xia_2013}.

When V_{deposition}/V_{diffusion} << 1, atoms deposited at the active region are able to diffuse quickly to the more stable sites, and thus the synthesis is under thermodynamic control.
The thermodynamic product is the nanocrystal shape with the minimum surface free energy.
The surface free energy is different for each crystallographic planes.
Among the low-index planes of an fcc metal, the surface free energies in vacuum increase in the order of (111) < (100) < (110).
The equilbrium shape of a nanocrystal in a vacuum can be theoretically predicted by the Wulff construction \cite{Bodineau_1999}.
In the synthesis of nanocrystals, capping agents such as ionic species, small molecules, and polymers can selectively bind to different facets and alter their specific surface free energies.
The final shape of the nanocrystal can be altered by capping agents through this mechanism if the synthesis is under thermodynamic control.

On the other hand when V_{deposition}/V_{diffusion} >> 1, most deposited atoms stay at the high surface energy region since they do not diffuse fast enough.
In this case, the synthesis is under kinetic control.
Kinetically controlled shapes depend on the relative deposition rates to different facets of the nanocrystal, which the shape can be predicted by the kinetic Wulff construction \cite{Zhang_2006}.
Various experimental parameters can influence the rates for atom deposition and surface diffusion.
The rate at which metal atoms are available for deposition directly correlates with V_{deposition}.
In the colloidal synthesis, typically metal atoms are supplied through the reduction of a salt precursor by a reductant.
Thus, the magnitude of V_{deposition} can be influenced by the concentration of reagents, reaction temperature, additives that form coordination complexes with the metal ion, and the injection rate of the precursor solution.
In contrast, surface diffusion is a thermally activated process where there is a potential energy barrier to diffusion.
The adatoms on the solid surface diffuses by a jumping or hopping mechanism \cite{1997}.
Consequently, V_{diffusion} is mainly determined by the reaction temperature and the height of the potential energy barrier.
The height of the potential energy barrier can be affected by factors such as the strength of bond between the surface atom and the adatom, the crystallographic plane of the surface, and capping agents surface passivation.

Experiments:

Cubes

Wires

PVP facet selectivity

PVP interactions

Bridge Tang 2008 Strain-induced

Simulations:

Fichthorn DFT

Fichthorn Force field

Solvent effects in experiments

Fichthorn solvent
  
  
  
  
  
  