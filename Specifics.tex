% Specifics
% Background info, context, limitations

The scope of my investigation is limited to the synthesis of colloidal metal nanocrystals, particularly silver (Ag).
The polyol synthesis is a popular solution-phase synthesis of Ag nanocrystals \cite{Skrabalak_2007}.
The typical reaction temperature is 150$^{\circ}$C.
In the polyol synthesis, ethylene glycol acts as both the solvent and the reducing agent.
The source of Ag is from silver nitrate that is dissolved in ethylene glycol.
Ag seeds may be added in order to disentangle growth from nucleation, enhancing control of nanocrystal size and shape.
Structure-directing agents, typically polyvinylpyrrolidone (PVP), are added to prevent aggregation of nanocrystals and to promote the formation of \{100\}-faceted nanocrystals.
The reaction conditions such as temperature, the concentration of silver nitrate, and the molar ratio of PVP to AgNO_3 is critical to the formation of different nanocrystal shapes.
These shapes can range from cubes \cite{Xia_2012,Zhang_2010}, triangular plates \cite{Lofton_2005,Liu_2012}, and five-fold twinned pentagonal wires \cite{Zhu_2011,Zhang_2008,Sun_2002}.
It was hypothesized that PVP promotes the formation of \{100\} facets by preferentially binding to \{100\} facets over \{111\} facets\cite{Xia_2012,Sun_2002}, also known as facet specific adsorption.

Binding of PVP to Ag surfaces can be characterized by the potential of mean force (PMF) profiles, which can be calculated by umbrella sampling \cite{Torrie_1977,K_stner_2011}.
The PMF represents the free energy of a system as the function of one collective variable.
Adsorption processes can be described by the free energy of the adsorbate as the function of the orthogonal distance from the surface where adsorption occurs.
Binding energies can be obtained from the difference between the PMF of the adsorbate at the adsorbing state and at the solvent phase.
In addition, kinetics of the adsorption process can be obtained from the energy barrier for adsorption in the PMF profile.
The limitations of this method are the collective variable must be accurately chosen to represent the physical nature of the system and the multidimensional energy landscape is reduced to one dimension thus it may not give the complete description of the system.
  
  
  
  
  
  
  
  
  
  