% Specifics
% Background info, history, context, limitations

The scope of my investigation is limited to the synthesis of colloidal metal nanocrystals, particularly silver (Ag).
The polyol synthesis is the most popular solution-phase synthesis of Ag nanocrystals \cite{Skrabalak_2007}.
The typical reaction temperature is $150^{\circ}$ C.
In the polyol synthesis, ethylene glycol acts as both the solvent and the reducing agent.
The source of Ag is from AgNO_3 that is dissolved in ethylene glycol.
Ag seeds may be added in order to disentangle growth from nucleation, enhancing control of nanocrystal size and shape.
Structure-directing agents, typically polyvinylpyrrolidone (PVP), are added to prevent aggregation of nanocrystals and to promote the formation of {100}-faceted nanocrystals.
The ratio of PVP to AgNO_3 is critical to the formation of different nanocrystal shapes, ranging from cubes \cite{Xia_2012,Zhang_2010}, triangular plates \cite{Lofton_2005,Liu_2012}, and five-fold twinned pentagonal wires \cite{Zhu_2011,Zhang_2008,Sun_2002}.
It was hypothesized that PVP promotes the formation of {100} facets by preferentially binding to {100} facets \cite{Xia_2012,Sun_2002}, also known as the facet specific adsorption phenomenon.